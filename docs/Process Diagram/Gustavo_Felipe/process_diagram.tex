\documentclass[12pt]{article}

\usepackage[portuguese]{babel}
\usepackage{graphicx}
\usepackage{float}

\usepackage{indentfirst}

\usepackage{geometry}

\geometry{
	paper=a4paper,
	margin=60pt
}

\author{
	Felipe Scherer Vicentin\\
	248283
	\and
	Gustavo Miller Santos\\
	248320
}

\title{Diagrama de Processo}

\begin{document}

\maketitle

\section*{Visão geral do diagrama}

\begin{figure}[H]
	\centering
	\includegraphics[width=\linewidth]{images/process_diagram.png}
\end{figure}

\section*{Explicação textual}

O processo foi dividido em planejamento, atuação cíclica (sprint), atuação constante (atividades guarda-chuva), e integração. A ideia foi de manter um processo cíclico com iterações curtas (duas semanas), para
incentivar entregas e replanejamentos frequentes. Aqui, partimos do pressuposto que o processo seria utilizado em um projeto como o da matéria de MC426, e portanto a equipe é responsável por todas
as etapas: especificação, desenvolvimento, validação e evolução.

\subsection*{Processo base}

O processo foi baseado na metodologia SCRUM: iterações curtas, com uma fase de planejamento, uma de desenvolvimento, e uma de integração a cada ciclo. Cada iteração consistirá de 2 semanas.

Similarmente ao SCRUM, o backlog é atualizado frequentemente independente dos ciclos. No entanto, o planejamento foi simplificado para incluir o refinamento (detalhamento e estimativa de esforço),
seguido de priorização, definição do que entra no ciclo seguinte, e por fim divisão de trabalho.

Feito isso, temos a sprint em si que consiste em desenvolvimento, criação e validação de testes, documentação, revisão da equipe e investigação de falhas.

Uma vez que a sprint chega ao fim, temos a etapa de integração: as mudanças são revisadas, os testes são executados, e é feita a release em produção.

\subsection*{Atividades fundamentais e guarda-chuva}

As quatro atividades fundamentais foram descritas acima. São elas:

\begin{itemize}
	\item \textbf{Especificação}: o backlog é atualizado constantemente à medida que a equipe interage para definir sua visão de longo prazo. Esse backlog é refinado na etapa de planejamento, momento em que são definidos os pormenores.
	\item \textbf{Desenvolvimento}: acontece dentro da sprint, e inclui a criação dos testes para garantia de qualidade do que foi desenvolvido.
	\item \textbf{Validação}: além dos testes, a equipe faz uma validação de cada feature a ser entregue. E ao final do ciclo, há novamente uma validação em conjunto.
	\item \textbf{Evolução}: a qualquer momento, as especificações podem ser alteradas pela equipe. Se isso acontecer, o backlog é atualizado, e as atividades serão repriorizadas para a sprint seguinte.
\end{itemize}

\subsection*{Principais artefatos}

Lorem ipsum dolor sit amet, officia excepteur ex fugiat reprehenderit enim labore culpa sint ad nisi Lorem pariatur mollit ex esse exercitation amet. Nisi anim cupidatat excepteur officia. Reprehenderit nostrud nostrud ipsum Lorem est aliquip amet voluptate voluptate dolor minim nulla est proident. Nostrud officia pariatur ut officia. Sit irure elit esse ea nulla sunt ex occaecat reprehenderit commodo officia dolor Lorem duis laboris cupidatat officia voluptate. Culpa proident adipisicing id nulla nisi laboris ex in Lorem sunt duis officia eiusmod. Aliqua reprehenderit commodo ex non excepteur duis sunt velit enim. Voluptate laboris sint cupidatat ullamco ut ea consectetur et est culpa et culpa duis.

\subsection*{Papéis dos envolvidos no processo}

Lorem ipsum dolor sit amet, officia excepteur ex fugiat reprehenderit enim labore culpa sint ad nisi Lorem pariatur mollit ex esse exercitation amet. Nisi anim cupidatat excepteur officia. Reprehenderit nostrud nostrud ipsum Lorem est aliquip amet voluptate voluptate dolor minim nulla est proident. Nostrud officia pariatur ut officia. Sit irure elit esse ea nulla sunt ex occaecat reprehenderit commodo officia dolor Lorem duis laboris cupidatat officia voluptate. Culpa proident adipisicing id nulla nisi laboris ex in Lorem sunt duis officia eiusmod. Aliqua reprehenderit commodo ex non excepteur duis sunt velit enim. Voluptate laboris sint cupidatat ullamco ut ea consectetur et est culpa et culpa duis.

\end{document}